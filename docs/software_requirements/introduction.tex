\section{Introduction}
% This section is used for multiple kinds of documents.  It describes the
% purpose of this document in the context of the rest of the product's
% documentation library.  


\subsection{Purpose}
% What is the purpose of this document as opposed to the vision and scope,
% software design document (492), testing plan (493), et al.

\subsection{Document Conventions}
% Describe the conventions used in this document.  For example, throughout this
% document, the use of the plural shall imply the singular unless otherwise
% stated.  Now you can avoid parenthetical plurals like student(s).   

\subsection{Intended Audience and Reading Suggestions}
% Formally state the intended audience for this document.  For the SRS, the
% audience is usually developers, quality control, and documentation.  You are
% free to describe whatever readers you feel are appropriate, but you should not
% describe the reader in terms of the class.  That is, do not refer to a teacher
% or other students.  

\subsection{Project Scope}
% Put this developed project in the context of the overall product.  This is a
% brief summary of the vision and scope.  Just enough for the reader to
% understand the context for the requirements in later sections.

\subsection{References}
% List other documents that you refer to in the rest of this document.  Include
% unreferenced, but important documents for the project.  


